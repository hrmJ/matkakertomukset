\documentclass[finnish]{standalone}\usepackage[]{graphicx}\usepackage[]{color}
%% maxwidth is the original width if it is less than linewidth
%% otherwise use linewidth (to make sure the graphics do not exceed the margin)
\makeatletter
\def\maxwidth{ %
  \ifdim\Gin@nat@width>\linewidth
    \linewidth
  \else
    \Gin@nat@width
  \fi
}
\makeatother

\definecolor{fgcolor}{rgb}{0.345, 0.345, 0.345}
\newcommand{\hlnum}[1]{\textcolor[rgb]{0.686,0.059,0.569}{#1}}%
\newcommand{\hlstr}[1]{\textcolor[rgb]{0.192,0.494,0.8}{#1}}%
\newcommand{\hlcom}[1]{\textcolor[rgb]{0.678,0.584,0.686}{\textit{#1}}}%
\newcommand{\hlopt}[1]{\textcolor[rgb]{0,0,0}{#1}}%
\newcommand{\hlstd}[1]{\textcolor[rgb]{0.345,0.345,0.345}{#1}}%
\newcommand{\hlkwa}[1]{\textcolor[rgb]{0.161,0.373,0.58}{\textbf{#1}}}%
\newcommand{\hlkwb}[1]{\textcolor[rgb]{0.69,0.353,0.396}{#1}}%
\newcommand{\hlkwc}[1]{\textcolor[rgb]{0.333,0.667,0.333}{#1}}%
\newcommand{\hlkwd}[1]{\textcolor[rgb]{0.737,0.353,0.396}{\textbf{#1}}}%
\let\hlipl\hlkwb

\usepackage{framed}
\makeatletter
\newenvironment{kframe}{%
 \def\at@end@of@kframe{}%
 \ifinner\ifhmode%
  \def\at@end@of@kframe{\end{minipage}}%
  \begin{minipage}{\columnwidth}%
 \fi\fi%
 \def\FrameCommand##1{\hskip\@totalleftmargin \hskip-\fboxsep
 \colorbox{shadecolor}{##1}\hskip-\fboxsep
     % There is no \\@totalrightmargin, so:
     \hskip-\linewidth \hskip-\@totalleftmargin \hskip\columnwidth}%
 \MakeFramed {\advance\hsize-\width
   \@totalleftmargin\z@ \linewidth\hsize
   \@setminipage}}%
 {\par\unskip\endMakeFramed%
 \at@end@of@kframe}
\makeatother

\definecolor{shadecolor}{rgb}{.97, .97, .97}
\definecolor{messagecolor}{rgb}{0, 0, 0}
\definecolor{warningcolor}{rgb}{1, 0, 1}
\definecolor{errorcolor}{rgb}{1, 0, 0}
\newenvironment{knitrout}{}{} % an empty environment to be redefined in TeX

\usepackage{alltt}
\usepackage{forest}
\usepackage[utf8]{inputenc}
\usepackage[T1]{fontenc}
\usepackage[finnish]{babel}
\IfFileExists{upquote.sty}{\usepackage{upquote}}{}


\usepackage[colorlinks]{hyperref}

\begin{document}



\begin{forest}
  for tree={
    child anchor=west,
    parent anchor=east,
    grow'=east,
  minimum size=1cm,%new possibility
  %text width=4cm,%
    %draw,
    anchor=west,
    edge path={
      \noexpand\path[\forestoption{edge}]
        (.child anchor) -| +(-5pt,0) -- +(-5pt,0) |-
        (!u.parent anchor)\forestoption{edge label};
    },
  }
[Topic indicators
    [dobj (28 \%)
        [Verbs of aquiring / searching: 55 \%
            [first person
                [78 \%
                    [\emph{{\color{red} Asunnon hankin yksityiseltä vuokranantajalta\, ja se oli valmiina saapuessani.}}, text width=7cm]
                    [\emph{Asumisen järjestin itselleni jo Suomesta käsin.}, text width=7.6cm]
                ]
            ]
            [third person / passive 22 \% ]
        ]
        [other: 45 \%
            [\emph{Yliopisto takaa asunnot kaikille vaihto-opiskelijoille.}, text width=7cm]
        ]
    ]
    [nmod (17 \%)
        [first person (38 \%)
            [\emph{{\color{red}Kuljin rautatieasemalta metrolla yliopiston asuntolalle\, missä...}}, text width=7.5cm]
        ]
        [other (62 \%)
            [locative cases (66 \%)
                [Elative
                    [\emph{Asunnosta muodostui kriittisin osa koko vaihtoa.}, text width=7cm]
                ]
                [Illative
                    [\emph{Suurin osa Pietarin vaihtareista majoittuu samaan asuntolaan}, text width=7cm]
                ]
                [Inessive
                    [\emph{Meidän lisäksi samassa asunnossa asui italialainen tyttö, jonka kanssa jaoimme vessan, suihkun ja jääkaapin.}, text width=7cm]
                ]
            ]
            [Asunnon suhteen / kanssa (20 \%)
                [\emph{{\color{red}Minulla kävi tuuri asunnon kanssa\, sillä eräs tuttuni omistaa asunnon Berliinissä.}}, text width=7.5cm]
            ]
            [other (14 \%)
                [\emph{Rekistöröityminen paikalliseksi asukkaaksi sujui helposti}, text width=7cm] 
            ]
        ]
    ]
    [root (16 \%)
        [asua + locative adjunct (86\%)
            [First person (93 \%)
                [\emph{{\color{red}Asuin vaihtoni ajan North Park Universityn kampuksella, Ohlson-asuntolassa.}}, text width=8cm]
            ]
            [other (5 \%)
                [\emph{Lähes kaikki Bratislavan kauppakorkeakoulun vaihto-opiskelijat asuivat Ekonom-nimisessä dormitoryssa.}, text width=8cm]
            ]
        ]
        [other: parsing errors]
    ]
    [nsubj (10 \%)
        [subject: asunto/asuntola (65\%) 
            [with copula (67 \%)
                [Possessive  (1st pers.)(64\%)
                    [\emph{Minulla oli valmiiksi asunto, kun saavuin Prahaan.}, text width =5cm]
                ]
                [Other (36\%)
                    [\emph{Asunto oli välittömästi valmiina käyttöön kun saavuin paikalle.}, text width = 5cm]
                ]
            ]
            [other (33 \%)
                [\emph{Asunto järjestyi erittäin helposti jo etukäteen yliopiston tarjoamassa asuntolassa.}, text width=8cm]
                [\emph{Asuntoni sijaitsi lähellä Takasakan juna-asemaa ja parin kilometrin päässä yliopistolta.}, text width=8cm]
            ]
        ]
        [subject: asuminen/asumisjärjestely/asuntoasia (21\%) 
            [\emph{Asuminen tapahtui kampuksella asuntoloissa kahden henkilön huoneissa.}, text width=7cm]
            [\emph{Asuminen maksaa vähemmän kuin Suomessa.}, text width=8cm]
        ]
        [other subjects (11 \%)]
    ]
    [nmod:gobj (10 \%)
        [verbs of aquiring / searching (100\%), text width=2cm
            [first person (12 \%)
                [\emph{Eniten jännitin asunnon saamista.}]
            ]
            [third person / passive (88 \%)
                [with copula (51 \%)
                    [\emph{Asunnon hankkiminen Kööpenhaminassa on erittäin hankalaa ja vuokrataso on hyvin korkea.}, text width=8cm]
                    [\emph{Asunnon hakeminen oli todella helppoa.}, text width=7cm]
                    [\emph{Kaikkein vaikein asia oli asunnon löytäminen.}, text width=7cm]
                ]
                [other (49 \%)
                    [helping (47 \%)
                        [\emph{Ammanin päästä kansainvälinen toimisto avusti asunnon etsimisessä.}, text width=4cm]
                    ]
                    [other (53 \%)
                        [\emph{Asunnon hankkiminen kannattaa aloittaa suhteellisen aikaisin.}, text width=4cm]
                        [\emph{Myös asunnon hankkiminen sujui minun kohdallani kivuttomasti.}, text width=4cm]
                    ]
                ]
            ]
        ]
    ]
    [nmod:poss (8 \%)
        [\emph{Asumisen taso on usein räikeästi huonompi kuin Suomessa, siihen kannattaa varautua jo alunperin.}, text width=8cm]
        [\emph{Jos halusi saada paikallisen opiskelija-asuntosäätiön asunnon, piti tehdä online-hakemus suoraan kyseiselle taholle.}, text width=8cm]
        [lot of parsing errors / similar to gmod]
    ]
    [nsubj:cop (5 \%)
        [subject: asuminen/asumisjärjestely/asumisratkaisu (33\%) 
            [\emph{Asuminen Tokiossa on aika kallista.}, text width=7cm]
        ]
        [subject:asunto/asuntola (59 \%)
            [\emph{Ulkomaisille opiskelijoille tarkoitettu asuntola on vain kolme vuotta vanha ja täten melko moderni.}, text width=7cm]
            [\emph{Yliopiston asuntola on mielestäni ihan toimiva asumisvaihtoehto vaihdon ajaksi.}, text width=8cm]
        ]
        [other subjects
            [\emph{Yleisesti ottaen Frankfurtin asuntotilanne on huono.}, text width=6cm]
        ]
    ]
    [other (5 \%)
    ]
]
\end{forest}

\end{document}
